\subsection{Relazioni Binarie}
\label{subsec:binrel}
Riportiamo la definizione di \emph{relazione binaria} \cite{bin_relations} su uno o due insiemi, che sarà utile per definire formalmente il concetto di \emph{grafo}, fondamentale all'interno di questo elaborato:
\begin{definition}
    Una \emph{relazione binaria} su $A,B$ è un sottoinsieme del prodotto cartesiano $A \times B$.\\
    Una \emph{relazione binaria} su $A$ è un sottoinsieme del prodotto cartesiano $A \times A$.\\
	Se $\mathcal{R}$ mette in relazione $u,v$, cioè $(u,v) \in \mathcal{R}$, si usa la notazione $u \mathcal{R} v$ o $\mathcal{R}(u,v)$.
\end{definition}
\begin{definition}
    L' \emph{insieme immagine} di un elemento $x$ dell'insieme $A$ attraverso la relazione $\mathcal{R}$ è l'insieme $\mathcal{R}(x) = \{y \in B \mid x \mathcal{R} y\}$.
\end{definition}
Alcune relazioni binarie dispongono di una o più delle seguenti caratteristiche:
\begin{definition}
    Sia $\mathcal{R}$ una relazione binaria su $A$. Siano $x,y,z$ qualsiasi appartenenti ad $A$. Allora $\mathcal{R}$ è:
    \begin{itemize}
        \item \emph{Riflessiva} se $x \mathcal{R} x$;
        \item \emph{Simmetrica} se $x \mathcal{R} y \implies y \mathcal{R} x$;
        \item \emph{Transitiva} se $(x \mathcal{R} y \land y \mathcal{R} z) \implies x \mathcal{R} z$.
    \end{itemize}
\end{definition}
\begin{example}
    La relazione ``$\leq$'' su $\mathbb{N}$ è riflessiva e transitiva, ma non simmetrica. La relazione ``$=$'' ($a = b \iff $``$a,b$ sono lo stesso numero'') su $\mathbb{N}$ è simmetrica, riflessiva e transitiva.
\end{example}
\begin{definition}\label{def:eq_rel}
    Una \emph{relazione di equivalenza} su un insieme $A$ è una relazione binaria riflessiva, simmetrica e transitiva. Si vede facilmente che questo genere di relazione partiziona $A$ in \emph{classi di equivalenza}, ovvero sottoinsiemi disgiunti di $A$ all'interno dei quali tutte le coppie di elementi sono in relazione.
\end{definition}
Data una relazione di equivalenza $\mathcal{R}$ su un insieme $A$, si usa la notazione ``$[a]_{\mathcal{R}}$'' per indicare la classe di equivalenza di $\mathcal{R}$ cui appartiene $\mathcal{R}$. Inoltre si usa la notazione ``$A/\mathcal{R}$'', letta ``\emph{quoziente} di $A$ rispetto a $\mathcal{R}$'', per denotare l'insieme delle classi di equivalenza di $\mathcal{R}$ su $A$.\\
In alcune situazioni risulta conveniente definire la più piccola relazione (cioè quella che mette in relazione il minor numero possibile di coppie) che dispone di una certa proprietà, e che contiene una relazione binaria di partenza. Una relazione costruita in questo modo è una ``\emph{chiusura}'':
\begin{definition}
	Sia $\mathcal{R}$ una relazione binaria su $A$. Le seguenti relazioni sono chiusure di $\mathcal{R}$:
    \begin{itemize}
        \item \emph{Riflessiva}: $\mathcal{R}_r = \mathcal{R} \cup \{(x,x) \mid x \in A\}$;
        \item \emph{Simmetrica}: $\mathcal{R}_s = \mathcal{R} \cup \{(y,x) \mid x \mathcal{R} y\}$;
        \item \emph{Transitiva}: $\mathcal{R}_t = \mathcal{R} \cup \{(x,z) \mid \exists y \in A,\, x \mathcal{R} y \land y \mathcal{R} z\}$.
    \end{itemize}
\end{definition}
\begin{example}
    La chiusura riflessiva della relazione ``$<$'' (minore stretto) è la relazione ``$\leq$''.
\end{example}
Nel seguito useremo ampiamente la definizione seguente:
\begin{definition}
    Sia $\mathcal{R}$ una relazione binaria su $A \times B$. La \emph{contro-immagine} di un elemento $y \in B$ rispetto ad $\mathcal{R}$ è l'insieme $\mathcal{R}^{-1}(y) = \{x \in A \mid x \mathcal{R} y\}$.\\
    Più in generale, la \emph{funzione inversa} di $R$ è la funzione $\mathcal{R}^{-1} : \mathcal{P}(B) \to \mathcal{P}(A)$ (dove ``$\mathcal{P}$'' denota l'insieme delle parti) che associa ad un sottoinsieme di $B$ tutti gli $x \in A$ tali che vale $x \mathcal{R} y$ per almeno un $y$ del sottoinsieme.
\end{definition}
Adottiamo infine la notazione ``$|A|$'' per indicare la \emph{cardinalità} dell'insieme $A$. In modo analogo, data una relazione binaria $\mathcal{R}$, $|\mathcal{R}|$ è il numero delle coppie messe in relazione da $\mathcal{R}$.
