\subsection{Sviluppi futuri}
Sebbene il pacchetto Python sia funzionante, testato e documentato, non si può dire che sia pienamente applicabile ad un gran numero di problemi pratici. Può essere utilizzato senza dubbio per la riduzione di grafi, ma vi sono alcune funzionalità critiche nella pratica che al momento non sono ancora implementate. Ne elenchiamo brevemente alcune, giustificando i motivi che ci portano a considerarne l'integrazione:
\begin{itemize}
    \item \emph{Labeled edges}: come abbiamo osservato nella Sezione \ref{sec:applitations} assegnare etichette agli archi consente di modellizzare problemi molto complessi ed interessanti, dunque integrare questa funzionalità aumenterebbe molto l'applicabilità del pacchetto;
    \item \emph{k-bisimulazione}: in alcuni casi la bisimulazione fornisce un partizionamento del grafo troppo approfondito, inutilizzabile all'atto pratico, poichè ciò che succede ``lontano'' da un nodo talvolta è di poca importanza per quel nodo; per questo motivo si introduce una variante, la k-bisimulazione, in cui le caratteristiche topologiche di un nodo (gli archi nella sua immagine, gli archi nell'immagine dei successori del nodo, \dots) vengono considerate solo in un ``intorno'' di raggio $k \geq 0$ \cite{kbisi}; la k-bisimulazione peraltro è piuttosto economica da calcolare, e trova quindi diverse applicazioni pratiche;
    \item Pianifichiamo infine di migliorare la compatibilità con altri package Python nell'ambito della teoria dei grafi, come ad esempio \emph{NetworkX}.
\end{itemize}

Intendiamo continuare lo sviluppo del pacchetto come progetto open source, possibilmente cercando di coinvolgere anche altri sviluppatori interessati all'argomento. Per quello che abbiamo potuto osservare non ci sono altri progetti Python che trattino la bisimulazione in modo approfondito.
