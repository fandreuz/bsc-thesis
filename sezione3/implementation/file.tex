\subsection{Implementazione}
Si è cercato di rendere il package funzionante su quante più versioni di Python 3 ci è stato possibile, rimpiazzando costrutti delle ultime versioni con features retrocompatibili.

Internamente grafi e partizioni sono rappresentati con classi apposite all'interno del file \verb|graph_entities.py|, di cui forniamo un breve glossario:
\begin{itemize}
    \item \verb|_Vertex|: un nodo del grafo;
    \item \verb|_Edge|: un arco del grafo;
    \item \verb|_QBlock|, \verb|_XBlock|: rappresentano rispettivamente i blocchi di tipo \verb|Q|,\verb|X| dell'algoritmo di Paige-Tarjan, negli algoritmi successivi solo la prima classe viene utilizzata, in quanto è più flessibile e contiene diverse funzioni utili (come \verb|_QBlock.mitosis(_Vertex[])|, che consente di dividere il blocco in due in tempo lineare, data una lista di nodi da estrarre).
\end{itemize}
Le classi contengono attributi utilizzati ampiamente dagli algoritmi per depositare informazioni utili.

Nel pacchetto sono stati implementati da zero i seguenti algoritmi, per il calcolo della massima bisimulazione o utilizzati come subroutine:
\begin{itemize}
    \item Algoritmo di Paige-Tarjan \ref{alg:pt};
    \item Algoritmo di Dovier-Piazza-Policriti \ref{alg:fba};
    \item Algoritmo incrementale di Saha \ref{alg:saha};
    \item Depth first search (varie versioni);
    \item Algoritmo di Kosaraju/Sharir \cite{sharir}.
\end{itemize}

Sono state utilizzate le seguenti librerie open source:
\begin{itemize}
    \item \emph{NetworkX}, per la rappresentazione dei grafi presi in input dall'utente;
    \item \emph{llist}, per un'implementazione delle \emph{doubly linked list}, fondamentali per una corretta stesura dell'algoritmo di Paige-Tarjan.
\end{itemize}
