\documentclass{beamer}
\usepackage{tikz}

%Information to be included in the title page:
\title{\texttt{BisPy}}
\subtitle{un pacchetto Python per il calcolo della massima bisimulazione di grafi diretti.}

\author{Francesco Andreuzzi}
\institute{Università degli Studi di Trieste,\\Dipartimento di Ingegneria e Architettura}
\date{14 Luglio 2021}

\begin{document}
\beamertemplatenavigationsymbolsempty

{\usebackgroundtemplate{%
    \parbox[c][\paperheight][c]{\paperwidth}{\centering \tikz\node[opacity=0.08] {\includegraphics[width=8cm,height=8cm]{../imgs/logo.png}};}}
    \begin{frame}
        \maketitle
        {\scriptsize Anno accademico 2020-2021 \hfill Relatore: Prof. Alberto Casagrande}
    \end{frame}
}

\begin{frame}
    \frametitle{Grafi}
    \begin{itemize}
        \item Cosa sono
        \item A cosa servono (modelli, social network)
    \end{itemize}
\end{frame}

\begin{frame}
    \frametitle{Massima bisimulazione}
    \begin{itemize}
        \item Definizione intuitiva
        \item Equivalenza di nodi
        \item Applicazioni
    \end{itemize}
\end{frame}

\begin{frame}
    \frametitle{Algoritmi}
    \begin{itemize}
        \item Paige-Tarjan
        \item Dovier-Piazza-Policriti
        \item Saha
    \end{itemize}
\end{frame}

\begin{frame}
    \frametitle{BisPy}
    \begin{itemize}
        \item Python
        \item Open source
    \end{itemize}
\end{frame}

\begin{frame}
    \frametitle{Risultati sperimentali}
\end{frame}

\begin{frame}
    \frametitle{Sviluppi futuri}
    \begin{itemize}
        \item Incremental delete
        \item Labeled edges
        \item NetworkX
        \item Cython
    \end{itemize}
\end{frame}

\begin{frame}
    Grazie per l'attenzione
\end{frame}

\end{document}
